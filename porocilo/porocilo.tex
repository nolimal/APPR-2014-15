\documentclass[11pt,a4paper]{article}

\usepackage[slovene]{babel}
\usepackage[utf8x]{inputenc}
\usepackage{graphicx}

\pagestyle{plain}

\begin{document}
\title{Poročilo pri predmetu \\
Analiza podatkov s programom R}
\author{Matevž Nolimal}
\maketitle

\section{Razmere na trgu dela doma in v EU}
Tabela za delovno aktivne po stopnjah dosežene izobrazbe, spolu in kohezijskih regijah, Slovenija, letno:\n
http://pxweb.stat.si/pxweb/Dialog/varval.asp?ma=0762104S&ti=&path=../Database/Dem_soc/07_trg_dela/02_07008_akt_preb_po_anketi/02_07621_akt_preb_ADS_letno/&lang=2\n
Tabela za brezposelne po stopnjah dosežene izobrazbe, spolu in kohezijskih regijah, Slovenija, letno:\n
http://pxweb.stat.si/pxweb/Dialog/varval.asp?ma=0762112S&ti=&path=../Database/Dem_soc/07_trg_dela/02_07008_akt_preb_po_anketi/02_07621_akt_preb_ADS_letno/&lang=2\n
Tabela za Zaposlene po regijah, spolu, času in ravni dosežene izobrazbe:\n
http://appsso.eurostat.ec.europa.eu/nui/submitViewTableAction.do\n

Tabela za ekonomsko aktivno populacijo po spolu, starosti, najvišji ravni pridobljene izobrazbe ter NUTR 2 regij(1000):\n
http://appsso.eurostat.ec.europa.eu/nui/show.do?dataset=lfst_r_lfp2acedu&lang=en\n


\section{Obdelava, uvoz in čiščenje podatkov}
12.11.2014 izbral Temo po predelavi podatkov na straneh Eurostat-a in SURS-a7, ter napisal poglavje tematika v README.md.

13.11.2014 v porocilo.tex napisal povezavi na tabele, ki sem si jih shranil v svoji mapi ter jih bom začel obdelovati\n
\n
14.11.2014 vnesel sem povezave do tabel v mapi podatki\n
\n
26.11.2014  vnesel sem program za uvoz podatkov CSV za zaposlenost v EU in sicer ZaposlenostEU\n
\n
27.11.2014  vnesel sem funkcijo, ki mi prebere podatke (po spremenljivkah)in izračuna povprečje\n
\n
28.11.2014  vnesel sem funkcijo, ki mi uvozi podatke CSV za AktivniSLO in urejenostno spremenljivk, ki primerja deleže in število aktivnih\n
\n
30.11.2014  vnesel sem funkcijo, ki mi uovzi podatke v obliki XML za AktivniSLO1\n
\n
1.12.2014 sem se ukvarjal s tipi spremenljivk v uvoženih XML datotekah\n
\n
2.12.2014 sem na vajah bolj optimalno spremenil tipe spremenljivk v tabelaxml1 in tabelaxml2\n
\n
3.12.2014 sem naredil grafe iz podatkov z uvoza\n
\n

\section{Analiza in vizualizacija podatkov}

\includegraphics{../slike/povprecna_druzina.pdf}

\section{Napredna analiza podatkov}

\includegraphics{../slike/naselja.pdf}

\end{document}
