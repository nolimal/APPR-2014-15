\documentclass[11pt,a4paper]{article}

\usepackage[slovene]{babel}
\usepackage[utf8x]{inputenc}
\usepackage{graphicx}
\usepackage{url}
\usepackage{pdfpages}
\usepackage{animate}

\pagestyle{plain}

\begin{document}

\begin{titlepage}

\newcommand{\HRule}{\rule{\linewidth}{0.5mm}}

\center

\textsc{\LARGE Fakulteta za matematiko in fiziko}\\[1.5cm]
\textsc{\Large Poročilo pri predmetu}\\[0.5cm]
\textsc{\large Analiza podatkov s programom R}\\[0.5cm]
\HRule \\[0.4cm]
{ \huge \bfseries Razmere na trgu dela doma in v EU}\\[0.4cm] 
\HRule \\[1.5cm]


\begin{minipage}{0.4\textwidth}
\begin{flushleft} \large
\emph{Avtor:}\\
Matevž \textsc{Nolimal}
\end{flushleft}
\end{minipage}
~
\begin{minipage}{0.4\textwidth}
\begin{flushright} \large
\emph{Mentor:} \\
Dr. Janoš \textsc{Vidali}
\end{flushright}
\end{minipage}\\[4cm]

{\large \today}\\[3cm] 
\vfill

\end{titlepage}


\section{Izbira teme}

Izbrana tematika obsega statistično obdelavo gibanja ravni zaposlenosti po večih spremenljivkah. Razsežnost problema vpliva ne le na denarne tokovi, temveč tudi na območja motiva bega možganov in iskanja višjega zadovoljstva posameznikov. S pomočjo podatkov zbranih na portalu Eurostat in SURS ter ustrezno metodologijo bom oblikoval trenutno in preteklo podobo stanja na trgu dela, opredeljeno na državni oziroma regionalni ravni znotraj Europskega obroča. 

Zaradi velikih odstopanj v regionalni podobi deleža in derivata zaposlenih, obstaja ogromno uporabnih tabel, ki bodo ključ moje obdelave zbranih podatkov. Pričakujem, da bom numerično potrdil znan Okunov zakon, ki pravi da višjo gospodarsko rast dosegamo ob nižji brezposelnosti, zato bom veliko časa namenil tudi obdelavi tega izredno pomembnega dejavnika. Podatki v obliki \verb|XSL|, \verb|CSV|, \verb|HTML| in \verb|PDF| bom obdelal po večih spremenljivkah, kot so npr. spol, starost, najvišja raven izobrazbe, regija itn. Posvetil se bom tudi stroškom dela, strukturi zaslužka, meri aktivnosti, saj vsi našteti pomembno vplivajo na končno podobo. 

Namen in cilj projekta je, da spoznam orodja analiziranja v programu R na konkretnem primeru in pridobim novo znanje s področja, ki ga bom analiziral. Dodana vrednost tega je interdisciplinarno povezovanje znanj, ki je zelo uporabno v modernih modelih poslovnega management-a. 
\pagebreak
\section{Obdelava, uvoz in čiščenje podatkov}

Po izbiri obravnavane teme sem se lotil zbiranja podatkov v mapi podatki iz strani \verb|Eurostat-a| in \verb|SURS-a|. Ko sem imel shranjene podatke v tabelah \verb|CSV| sem v podprogramu \verb|uvoz.r| uvozil podatke iz datotek vrste \verb|CSV| in datotek vrste \verb|XML| (potrebno je bilo dodatno naložiti tudi paket \verb|XML|). Po uvozu sem v vsako izmed teh tabel dodal po vsaj eno urejenostno spremenljivko (ta je ponavadi primerjala podatke po vrsticah in jih nato rangirala). 

Ker je bila uvožena tabela \verb|ZaposlenostEU| izredno velika in njeni podatki dokaj zahtevni za obdelavo sem si ustvaril pomožno matriko \verb|povprecje|, s katero sem si pomagal rangirati podatke po vrsticah in iz tega poračunati povprečje (to bom uporabil v kasnejših fazah). 

Pri vsem tem delu sem veliko časa namenil tipom spremenljivk. Ob uvozu preko \verb|read.csv2| sem naletel na težave s tipom spremenljivk factor, ki sem jih nato spremenil bodisi v character bodisi v integer. Pri uvozu preko \verb|XML| paketa pa sem dobil rezultate v tipu double, ki sem jih spremenil v integer in kjer je bilo potrebno v character. Uvožene podatke sem shranil v tabelah \verb|tabelaxml1| in \verb|tabelaxml2|. 

Po zaključenem uvozu podatkov sem jih začel smiselno urejati in prikazati v grafih. Ustvarjanje grafov iz podatkov uvoza mi je vzelo veliko časa, saj sem zaradi kompleksnosti prikazanih podatkov potreboval veliko časa (pred\-vsem se to nanaša na \verb|grafplot.pdf|). Po prejeti vmesni oceni sem se lotil še popravljanja potrebnih zadev (poročilo in glavnega programa \verb|projekt.r|, ki do tedaj ni imel nobenega pomena). Na naslednji strani se nahaja animacija z vsemi vključenimi grafi iz druge faze. Spodaj je podan opis le teh.
% \makebox[\textwidth][c]{
% 
% \includegraphics[width=1.35\textwidth]{../slike/grafplot.pdf}
% 
% }
% 
% Zgornji graf - \verb|grafplot.pdf| prikazuje šte\-vilo aktivnih v Sloveniji (v tisočih) glede na doseženo raven izobrazbe in ločeno po letih. 
% 
% \makebox[\textwidth][c]{
% 
% \includegraphics[width=1.35\textwidth]{../slike/grafpoklici.pdf}
% 
% }
% 
% Zgornji graf - \verb|grafpoklici.pdf| prikazuje višino povprečne mesečne bru\-to plače odvisne od poklica za Slovenijo v letu 2013. 
% 
% \makebox[\textwidth][c]{
% 
% \includegraphics[width=1.35\textwidth]{../slike/grafsektorji.pdf}
% 
% }
% 
% Zgornji graf - \verb|grafsektorji.pdf| prikazuje višino povprečne mesečne bru\-to plače odvisne od sektorja zaposlitve za Slovenijo v letu 2013. 
% 
% \makebox[\textwidth][c]{
% 
% \includegraphics[width=1.35\textwidth]{../slike/graftorta.pdf}
% 
% }
% 
% Zgornji graf - \verb|graftorta.pdf| prikazuje tortni diagram višine plač po sektorjih (velikost kosa nam pove povprečno višino plače). Ta diagram nima nekega posebnega pomena - gre zgolj za nov primer uporabnega prikaza podatkov s programom R. 

\begin{center}
\animategraphics[controls,loop, width=1\linewidth]{1}{graf}{0}{4}\\
\end{center}

Prvi graf - \verb|Plače po sektorjih| prikazuje višino povprečne mesečne bru\-to plače odvisne od sektorja zaposlitve za Slovenijo v letu 2013. Po podatkih sodeč so te najvišje v javnih družbah. Javne družbe so izdajatelji, katerih vrednostni papirji so uvrščeni v trgovanje na organiziranem trgu v Republiki Sloveniji ali v drugi državi članici Evropske Unije. Za njih velja posebna obveznost razkrivanja nadzorovanih informacij (ZTFI). Med njimi so tudi NLB, KD skupina, Pivovarna Laško, Cinkarna, Žito, Triglav naložbe in še mnoge druge. Javne družbe so pomemben del gospodarstva Slovenije, saj ima SLovenija eden izmed največjih deležov gospodarstva v svoji lasti v primerjavi z drugimi članicami Evropske Unije. Kako dobro, če sploh, je to za gospodarstvo sem deloma analiziral tudi v četrti fazi. Na drugi strani pa vidimo, da po podatkih SURS-a najmanjše bruto plače prejemajo zaposleni v zasebnem sektorju, toda zavedati se moramo, da so tukaj obeleženi le redni prejemki.

Drugi graf - \verb|plače po poklicih| prikazuje višino povprečne mesečne bru\-to plače odvisne od poklica za Slovenijo v letu 2013. Iz grafa je razvidno, da je povprečna bruto plača v Sloveniji približno 1600 \verb|EUR| mesečno. Zanimivo pa je, kako se to povprečje oblikuje po raznih poklicnih skupinah. Opazimo, da največje prejemke pridobivajo poklici v skupini Zakonodajalci, visoki uradniki in menedžerji, na drugi strani pa pričakovano najmanjše prejemajo zaposleni v okviru Poklici za preprosta dela.

Tretji graf - \verb|Tortni diagram plač po sektorjih| prikazuje tortni diagram višine plač po sektorjih (velikost kosa nam pove povprečno višino plače). Ta diagram nima nekega posebnega pomena - gre zgolj za nov primer uporabnega prikaza podatkov s programom R.

Četrti graf - \verb|Aktivnost v Sloveniji| prikazuje šte\-vilo aktivnih v Sloveniji (v tisočih) glede na doseženo raven izobrazbe in ločeno po letih. Ob omenjenemu grafu lahko zaradi vsesplošne ekonomske situacije med leti 2007 in 2013 prvič pomislimo o morebitni povezavi med gospodarsko rastjo in delovno akvtinostjo. V mislih imam tako imenovan Okunov zakon, ki sem da v delu obsežne analize tudi dokazal. Sestavil sem tudi napovedni model tako za Slovenijo, kot za države Evropske Unije. Na četrtem grafu v animaciji lahko opazimo tudi trende. Opazno dejstvo je manjšanje delovnih mest za manj izobražene v primerjavi s populacijo z višješolsko oziroma visokošolsko stopnjo izobrazbe. Nekvalificirana sila je v težkih gospodarskih situacijah prva, kjer spremembe v finančni strukturi podjetij direktno vplivajo na zaposlenost. Torej gotovo se v času krize splača dlje študirati in pridobiti višjo izobrazbo za ceno morebitne zaposlitve v prihodnosti.

\pagebreak
\section{Analiza in vizualizacija podatkov}

Za analizo in vizualizacijo podatkov sem sprva vpeljal zemljevid sveta-\verb|svet|, v katerem so zbrani mnogi uporabni podatki, \verb|pop_est| sem uporabil tudi na zemljevidu držav EU. Ker sem obdeloval zgolj podatke zbrane iz držav članic EU, sem iz \verb|svet| pobral potrebne podatke in jih združil v \verb|EU|. Ker moje delo analizira razmere na trgu dela sem uvozil novo \verb|csv| datoteko z naslovom \verb|NezaposlenostEU|. Ker sem za obdelavo podatkov potreboval točno takšen vrstni red, kot pri tabeli \verb|EU|, sem poskrbel za popolno ujemanje vrstičnih imen, zato funkcije \verb|preuredi| sploh nisem potreboval, toda sem delal neposredno z indeksi. Tako sem spremenil vrstni red in dobil \verb|NNezaposlenostEU|. Pri vsem tem je bilo potrebno poskrbeti za pretvorbo v nize, saj so imena v zemljevidih predstavljena kot faktorji. 

Z ukazom \verb|spplot| in \verb|plot| sem ustvaril naslednje zemljevide: 
\begin{enumerate} 
\item{\verb|EUSkupno.pdf|}

\item{\verb|EUZenske.pdf|}

\item{\verb|EUMoski.pdf|}

\item{\verb|prebivalstvo.pdf|}
\end{enumerate}

Prve tri tabele vsebujejo podatke iz uvožene \verb|csv| datoteke \texttt{Ne\-za\-pos\-le\-nost\-EU}, zadnja pa iz zemljevida \verb|EU|. 

Po zaključeni obdelavi podatkov sem jih začel smiselno urejati in prikazati na zemljevidih. Ustvarjanje zemljevidov iz urejenih podatkov uvoza mi je vzelo dosti časa, saj sem moral popraviti tudi koordinate in smiselno razdeliti zemljevid, da je prikazal potrebno.Na naslednjih straneh se nahajajo omenjeni zemljevidi. 

\makebox[\textwidth][c]{

\includegraphics[width=1.35\textwidth]{../slike/EUSkupno.pdf}

}

Zgornji zemljevid - \verb|EUSkupno.pdf| prikazuje delež nezaposlenih med aktivnimi prebivalci v letu 2013 v večini držav članic Evropske Unije.Podatki so iz uvožene \verb|csv| datoteke z naslovom \verb|NezaposlenostEU|. Opazimo, da smo na žalost med državami z največjo stopnjo nezaposlenosti med državami Evropske Unije. Usodo si delimo še z Litvo in Španijo. Najmanjšo stopnjo brezposelnosti pa so v letu 2013 imeli po podaktih Eurostat-a na Finskem.

\makebox[\textwidth][c]{

\includegraphics[width=1.35\textwidth]{../slike/EUZenske.pdf}

}

Zgornji zemljevid - \verb|EUZenske.pdf| prikazuje delež nezaposlenih med aktivnimi prebivalkami v letu 2013 v večini držav članic Evropske Unije. Podatki so iz uvožene \verb|csv| datoteke z naslovom \verb|NezaposlenostEU|. Opazimo, da imajo ženske največjo stopnjo brezposelnosti v Litvi. Slovenija ima tudi eno izmed višjih stopenj brezposelnosti med članicami Evropske Unije za ženske. Najnižjo stopnjo registrirane brezposelnosti pa imajo na Finskem.

\makebox[\textwidth][c]{

\includegraphics[width=1.35\textwidth]{../slike/EUMoski.pdf}

}

Zgornji zemljevid - \verb|EUMoski.pdf| prikazuje delež nezaposlenih med aktivnimi moškimi prebivalci v letu 2013 v večini držav članic Evropske Unije. Podatki so iz uvožene \verb|csv| datoteke z naslovom \verb|NezaposlenostEU|. Slovenija je na tem področju po podatkih Eurostat-a na samem dnu Evropske Unije, toda zavedati se moramo dejstva, da registrirana brezposelnost ni ekvivalentna anketni brezposelnosti. Lahko gre zgolj za dejstvo, da se v Sloveniji mnogim izplača biti raje brezposeln in ponuditi svoje delo v sivi ekonomiji. Ker imamo za brezposelne kar dobre socialne transferje, lahko ravno obraten pojav vidimo v Rusiji, kjer je registrirana brezposlenost nizka, saj se prebivalcem ne izplača biti brezposeln zaradi slabih socialnih prejemkov.

\makebox[\textwidth][c]{

\includegraphics[width=1.35\textwidth]{../slike/prebivalstvo.pdf}

}

Zgornji zemljevid - \verb|prebivalstvo.pdf| prikazuje ocenjeno velikost prebivalstva v obravnavanih državah Evropske Unije za leto 2013. Slednja ocena je ključna za boljšo predstavo deležov v prejšnjih zemljevidih. Podatki so iz uvoženega zemljevida \verb|svet|. 

\pagebreak
\section{Napredna analiza podatkov}

Začetni problem mojega projektnega dela je bil dokazati tako imenovani Okunov zakon, ki izpostavlja pomen agregatne proizvodnje na trgu dela. Izpostavlja linearno povezanost med naraščajočo zaposlenostjo oziroma padajočo nezaposlenostjo v odvisnosti od pozitvne gospodarske rasti. 

V prvih fazah sem uvozil večino potrebnih podatkov o deležu nezaposlenih oz. neaktivnih po posameznih državah EU. Dodatno sem razčlenil le podatke za Slovenijo, saj me je zanima ali se tudi slednja obnaša skladno z osnovnim makroekonomskim zakonom.

V 4.fazi sem z namenom poiskati povezavo potreboval še podatke o bruto domačem proizvodu držav članic EU, ki sem jih našel na strani Eurostat-a v obliki \verb|CSV| datotek. Prišel sem do naslednjih zaključkov, ki jih bom sprva predstavil v obliki grafov.

\makebox[\textwidth][c]{

\includegraphics[width=1.35\textwidth]{../slike/EUOkunl.pdf}

}
Zgornji graf povezanosti kaže negativno linearno zvezo med nezaposlenostjo in gospodarsko rastjo. Točke na grafu predstavljajo članice EU, modra premica z začetno vrednostjo 11.9 in koeficientom -0.74. Omenjena podatka sem ugotovil iz povezanosti, ki je razvidna po spremenljivki \verb|lin|, ki je nastala s parjenjem oziroma iskanjem linearne zveze med tema spremenljivkama. Ta premica oziroma zveza je v makroekonomskem svetu poznana pod imenom Okunov zakon. Točke v levem spodnjem kotu, ki se najmanj ujemajo s pričakovanim so Švica, Norveška in Združeno kraljestvo. Razlog, da te države najbolj odstopajo lahko iščemo v mentaliteti prebivalstva, saj prebivalci ZK veljajo za konzervativne in so se že večkrat v zgodovini zoperstavili samoumevnemu odzivanju na spremembe v finančnem svetu(npr. večna dilema tečaja funt-evro). Razlog manjšega odzivanja Norveške in Švice je gotovo v tem, da je njihov bruto domači proizvod tako visok, da lažje prenese finančne šoke na trgu za ceno manjšega povzročanja skrbi med prebivalci. Ob enem je bil to tudi čas stabilnega deviznega tečaja med evrom in švicarskimi franki.Torej čeprav sem za omenjen graf potreboval le podatke o deležu nezaposlenih in gospodarsko rastjo med letom 2012 in 2013, sem dokazal tisto, kar sem si zadal že v prvi fazi projekta.

Na naslednjem grafu sem pokazal še nelinearno zvezo oziroma kvadratno zvezo, t.j. kvadratno funkcijo, ki se najbolj prilega podatkom. Tudi ta potrjuje obravnavani zakon.Sedaj lahko s funkcijo \verb|predict| ugotovim pričakovano stopnjo zaposlenosti. Tako sem dosegel cilj prvega dela četrte faze.

\makebox[\textwidth][c]{

\includegraphics[width=1.35\textwidth]{../slike/EUOkunk.pdf}

}

\pagebreak
V drugem delu četrte faze sem se osredotočil zgolj na trg dela v odvisnosti od agregatne proizvodnje v Sloveniji med leti 2008 in 2013, saj me je ob vseh teh podatkih zanimalo ali se je tako kot večina držav znotraj Unije obnašala tudi Slovenija. 

Odgovor za to navidez preprosto vprašanje še zdaleč ni tako enoznačen, kot je to veljajo za večino ostalih držav Evropske Unije. Najlažje bo idejo povezanosti razčleniti ob naslednjih grafih, ki prikazujejo odvisnost brezposelnosti od rasti bruto domačega proizvoda.

\makebox[\textwidth][c]{

\includegraphics[width=1\textwidth]{../slike/EUOkunsl.pdf}

}
Na zgornjem grafu lahko potrdimo Okunov zakon v manjši meri, kot na podlagi podatkov, ki so zajemali celotno statistiko Evropske Unije. Najbolj zanimivo je iskati vzroke za dejstvo, da se makroekonomski zakon ne potrjuje povsem suvereno. Največje odstopanje od povprečne vrednosti lahko zasledimo pri podatku za leto 2009, ko je slovensko gospodarstvo utrpelo skoraj 8 odstotni padec namesto od želene 2-3 odstotne rasti. Kje tilijo razlogi za kar dvakrat večji padec od povprečja Evropske Unije v letu 2009, kako je obseg državnega proračuna padel za kar dve milijardi evrov v enem letu.

Prva možna razlaga je slab odziv politike na finančno krizo. Z drugačnimi monetarnimi oziroma fiskalnimi politikami, bi depresijo lahko prebrodili veliko bolj učinkovito. Druga opcija je lahko zadolženost slovenskih podjetij (velik delež državnega lastništva) zaradi investicijskih projektov, ki v času krize niso prinašale dovolj velikega donosa v primerjavi s stroški premoženja. Tretji razlog pa bi lahko bil dolgoletna posledica neracionalnosti v gospodarstvu zaradi odlašanja premika k bolj tehnološkim panogam, torej reševanje "izgubljenih" podjetij je omililo odziv na brezposelnost. Omenjena bi po Okunovem zakonu morala v prihodnjih obdobjih poskočiti bolj kot je razvidno na grafu. To so trije poglavitni razlogi za odstopanje od pričakovanega večjega naklona (ta je tukaj le -0.03).

\makebox[\textwidth][c]{

\includegraphics[width=1.35\textwidth]{../slike/EUOkunsn.pdf}

}

Na zgornjem grafu sem prikazal še metodo prileganja z zlepki. Za slednjo sem moral naložiti paket \verb|mgcv|, da sem lahko uporabil funkcijo \verb|loess|.


\pagebreak
\section{Informacije o uporabljenih tabelah}

V tem razdelku bom opisal pomembnejše tabele, ki sem jih uporabil v projektu, ter tipe njihovih spremenljivk.

Tabela \verb|ZaposlenostEU| vsebuje 96 opazovanj za 30 spremenljivk. Spremenljivke so naslednje:

\begin{itemize}
  \item V stolpcu \textit{Država} se nahaja ime države. Spremenljivka je imenska.
  \item V stolpcu \textit{Spol} so podani Total/Males/Females, ki so prav tako imenska spremenljivka.
  \item V stolpcih \textit{Skupno.2008} do \textit{Skupno.2013} sestavljajo podatki o skupnem številu aktivnih v milijonih  za posamezne držav ločeno po spolu. Spremenljivka je številska.
  \item V stolpcih \textit{Ravni.0.2.2008} do \textit{Ravni.0.2.2013} in podobno še za ostale ravni do \textit{Ravni.5.8.2013} sestavljajo podatki o številu aktivnih z do\-se\-že\-no izobrazbo ravni po ISCED kategorizaciji v milijonih za posamezne držav ločeno po spolu. Spremenljivke so številske.  
  \item Stolpec \textit{Primerjava.1}, ki sem ga sam dodal, primerja podatke med leti 2008 in 2013 in jih razvrsti bodisi Več ali Manj. Ta spremenljivka je urejenostna. Podobno delujejo tudi \textit{Primerjava.2}, \textit{Primerjava.3} in \textit{Primerjava.4}.
\end{itemize}

Tabela \verb|AktivniSLO| vsebuje 54 opazovanj za 17 spremenljivk. Spremenljivke so naslednje:

\begin{itemize}
  \item V stolpcu \textit{Regija} se nahaja ime regije. Spremenljivka je imenska.
  \item V stolpcu \textit{Spol} so podani Skupno/Moški/Ženske, ki so prav tako imenska spremenljivka.
  \item V stolpcu \textit{Izobrazba} je informacija o zaključeni izobrazbi opazovanih objektov. Kategorij je pet. Spremenljivka je imenska.
  \item V stolpcih \textit{Aktivni.08} do \textit{Aktivni.13} so podatki o številu aktivnih z doseženo izobrazbo ravni v tisočih za posamezne regije, ločeno po spolu. Spremenljivke so številske.
   \item V stolpcih \textit{Delež.aktivnih.08} do \textit{Delež.aktivnih.13} so podatki o deležih aktivnih z doseženo izobrazbo ravni v tisočih za posamezne regije, ločeno po spolu. Spremenljivke so številske.
   \item Stolpec \textit{Primerjava.08.13}, ki sem ga sam dodal, primerja podatke med leti 2008 in 2013 in jih razvrsti bodisi več ali manj. Ta spremenljivka je urejenostna. Podobno deluje tudi \textit{Primerjava.deleža.08.13}.
\end{itemize}

Tabela \verb|tabelaxml1| vsebuje 11 opazovanj za 5 spremenljivk. Spremenljivke so naslednje:

\begin{itemize}
  \item V stolpcu \textit{Poklicne skupine} se nahaja ime poklicne skupine. Spremenljivka je imenska.
  \item V stolpcu \textit{Skupaj} so podani podatki o povprečni bruto plači za oba spola. Gre za številsko spremenljivko.
  \item V stolpcu \textit{Moški} so podani podatki o povprečni bruto plači za moške. Gre za številsko spremenljivko.
  \item V stolpcu \textit{Ženske} so podani podatki o povprečni bruto plači za ženske. Gre za številsko spremenljivko.
   \item V stolpcu \textit{Razmerje Ž/M} je podano razmerje med plačo po istih skupinah poklicev ločeno po spolu.
\end{itemize}

Tabela \verb|tabelaxml2| vsebuje 5 opazovanj za 4 spremenljivke. Spremenljivke so naslednje:

\begin{itemize}
  \item V stolpcu \textit{row.names} se nahaja ime sektorja. Spremenljivka je imenska.
  \item V stolpcu \textit{Skupaj} so podani podatki o povprečni bruto plači po sektorju. Gre za številsko spremenljivko.
  \item V stolpcu \textit{Osnovnošolska izobrazba ali manj} so podani podatki o povprečni bruto plači. Gre za številsko spremenljivko.
  \item V stolpcu \textit{Sprednješolska izobrazba} so podani podatki o povprečni bruto plači. Gre za številsko spremenljivko.
   \item V stolpcu \textit{Visoka, višja izobrazba} so podani podatki o povprečni bruto plači. Gre za številsko spremenljivko.
\end{itemize}

Tabela \verb|GdpEU| vsebuje 29 opazovanj za 6 spremenljivk. Spremenljivke so naslednje:

\begin{itemize}
  \item V stolpcu \textit{row.names} se nahaja ime države. Spremenljivka je imenska.
  \item V stolpcu \textit{Leto} je podana spremnljivka po letih. Gre za številsko spremenljivko.
  \item V stolpcu \textit{Trenutno} so podani podatki o BDP-ju držav članic EU za leto 2012 v EUR. Gre za številsko spremenljivko.
  \item V stolpcu \textit{Predhodno} so podani podatki o BDP-ju držav članic EU za leto 2013  EUR. Gre za številsko spremenljivko.
  \item V stolpcu \textit{Sprememba} je podana sprememba med letom 2012 in 2013. Gre za številsko spremenljivko.
  \item V stolpcu \textit{Rast} so podani podatki o rasti BDP za omenjeno obdobje. Gre za številsko spremenljivko.
  \item V stolpcu \textit{Nezaposleni} je podan delež nezaposlenih po državah članicah v letu 2013.
\end{itemize}

Tabela \verb|GdpEU| vsebuje 240 opazovanj za 6 spremenljivk. Spremenljivke so naslednje:

\begin{itemize}
  \item V stolpcu \textit{country} se nahaja ime države. Spremenljivka je imenska.
  \item V stolpcu \textit{capital} se nahaja ime prestolnice. Spremenljivka je imenska.
  \item V stolpcu \textit{lat} se nahaja geometrijska širina. Gre za številsko spremenljivko.
  \item V stolpcu \textit{long} se nahaja geometrijska dolžina. Gre za številsko spremenljivko.
  \item V stolpcu \textit{CountryCode} je podana okrajšava za državo. Spremenljivka je imenska.
  \item V stolpcu \textit{ContinentName} se nahaja ime kontinenta. Spremenljivka je imenska.
\end{itemize}

Tabela \verb|svet| vsebuje 177 opazovanj za 63 spremenljivk. Pomembnejše spremenljivke, ki sem jih večkrat uporabil so naslednje:

\begin{itemize}
  \item V stolpcu \textit{name long} se nahaja ime države. Spremenljivka je imenska.
  
  \item V stolpcu \textit{pop est} je ocenjeno število prebivalcev. Gre za številsko spremenljivko
  
  \item V stolpcu \textit{continent} se nahaja ime kontinenta. Spremenljivka je imenska.
\end{itemize}

Iz tabele \verb|svet| sem sestavil mnogo podtabel, ki sem jim dodajal razne podatke iz že omenjenih tabel. Tako so nastale tabele: \verb|EU|, \verb|EU1|, \verb|Evropa|, \verb|Evropa1| in še mnoge druge. Za vizualizacijo, je bila gotovo ena izmed najbolj pomembnih ravno omenjena tabela \verb|svet|.

\pagebreak
\section{Informacije o virih}

\begin{thebibliography}{9}

\bibitem{1}
  \url{http://appsso.eurostat.ec.europa.eu/nui/show.do?dataset=lfst_r_lfsd2pop&lang=en},
  {Accessed: 15-02-2015}
\bibitem{2}
  \url{http://appsso.eurostat.ec.europa.eu/nui/show.do?dataset=lfst_r_lfe2emp&lang=en},
  {Accessed: 15-02-2015}
\bibitem{3}
  \url{http://appsso.eurostat.ec.europa.eu/nui/show.do?dataset=lfst_r_lfu3pers&lang=en},
  {Accessed: 15-02-2015}
\bibitem{4}
  \url{http://pxweb.stat.si/pxweb/Database/Dem_soc/07_trg_dela/02_07008_akt_preb_po_anketi/01_07620_akt_preb_ADS_cetrt/01_07620_akt_preb_ADS_cetrt.asp},
  {Accessed: 15-02-2015}
\bibitem{5}
  \url{http://pxweb.stat.si/pxweb/Database/Dem_soc/Dem_soc.asp#07},
  {Accessed: 15-02-2015}
\bibitem{6}
  \url{http://www.stat.si/novica_prikazi.aspx?id=6779}
  {Accessed: 15-02-2015}
\bibitem{7}
  \url{http://www.stat.si/novica_prikazi.aspx?id=6777}
  {Accessed: 15-02-2015}
\bibitem{8}
  \url{http://www.a-tvp.si/default.aspx?id=93}
  {Accessed: 15-02-2015}
\bibitem{9}
  \url{https://www.google.si/webhp?sourceid=chrome-instant&ion=1&espv=2&ie=UTF-8#q=okunov}
  {Accessed: 15-02-2015}
\bibitem{10}
  \url{http://www.naturalearthdata.com/http//www.naturalearthdata.com/download/110m/cultural/ne_110m_admin_0_countries.zip}
  {Accessed: 15-02-2015}
  
\end{thebibliography}
\end{document}